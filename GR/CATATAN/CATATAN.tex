\documentclass[a4paper,11pt]{extarticle}
\usepackage[a4paper]{geometry}
\geometry{verbose,tmargin=2cm,bmargin=2cm,lmargin=2cm,rmargin=2cm}

\usepackage{fontspec}
\setmonofont{FreeMono}

\setlength{\parindent}{0cm}
\setlength{\parskip}{0.5em}

\usepackage{textcomp}

\usepackage{hyperref}
\usepackage{url}
\usepackage{xcolor}

\usepackage{minted}
\newminted{c}{breaklines,fontsize=\small}
\newminted{text}{breaklines,fontsize=\small}

\definecolor{mintedbg}{rgb}{0.95,0.95,0.95}
\usepackage{mdframed}

\BeforeBeginEnvironment{minted}{\begin{mdframed}[backgroundcolor=mintedbg]}
\AfterEndEnvironment{minted}{\end{mdframed}}

\title{
Pengenalan GR Framework
}
\author{Fadjar Fathurrahman}
\date{2018}

\begin{document}
\maketitle

\section{Contoh pertama}
\begin{ccode}
#include <stdio.h>
#include <stdlib.h>
#include <math.h>
#include <gr.h>

int main()
{
  const int N = 10;
  double *x, *y;
  x = (double*)malloc(N*sizeof(double));
  y = (double*)malloc(N*sizeof(double));
  
  const double XLEFT=0.0, XRIGHT=1.0;
  double dx = (XRIGHT - XLEFT)/(N-1);

  int i;
  for(i = 0; i < N; i++) {
    x[i] = XLEFT + i*dx;
    y[i] = sin(x[i])*cos(x[i]);
  }

  gr_polyline(N, x, y);
  gr_axes(gr_tick(0,1), gr_tick(0,1), 0, 0, 1, 1, -0.01);

  getc(stdin);
  return 0;
}
\end{ccode}

Set output dalam bentuk PDF (name file default \texttt{gks.pdf})

\textbf{TODO}: Bagaimana mengubah nama output menjadi selain dari \texttt{gks.pdf}?



\end{document}